\input{preamble}

% OK, start here.
%
\begin{document}

\title{C.4}


\maketitle

\phantomsection
\label{section-phantom}


\tableofcontents


\section{Introduction}
\label{section-introduction}

\noindent
We try to understand how to go from a groupoid schemne to a
Tannakian category and then back to the groupoid.
A reference is \cite{Deligne-tannakian}.



\section{Descent lemma for quasi-coherent modules}
\label{section-descent}

\begin{lemma}
\label{lemma-descent}
Let $(U, R, s, t, c)$ be a groupoid scheme. Let $g : U' \to U$ be a morphism
of schemes. Let $(U', R', s', t', c')$ be the restriction of $(U, R, s, t, c)$
by $g$. The pullback functor
$$
\QCoh(U, R, s, t, c)
\longrightarrow
\QCoh(U', R', s', t', c')
$$
is an equivalence if the morphism
$$
\xymatrix{
T = U' \times_{g, U, t} R \ar[r]_-{\text{pr}_1} \ar@/^3ex/[rr]^h
& R \ar[r]_s & U
}
$$
is quasi-compact, flat, and surjective.
\end{lemma}

\begin{proof}
We construct a functor quasi-inverse to the pullback functor.
Thus given an object $(\mathcal{F}', \alpha')$ of
$\QCoh(U', R', s', t', c')$ we have to construct an object
of $\QCoh(', R, s, t, c)$. To do this we can try to descend
$(T \to U')^*\mathcal{F}'$ to a quasi-coherent module on $U$.
Since $\{T \to U\}$ is an fpqc covering, it suffices for this
to construct an isomorphism of the two pullbacks of $\mathcal{F}'$
to $T \times_U T$ satisfying the cocycle conditions on
$T \times_U T \times_U T$. For this, since we have $\alpha'$, it suffices
to construct a morphism
$$
T \times_U T =
(U' \times_{g, U, t} R)
\times_U
(U' \times_{g, U, t} R)
\longrightarrow
(U' \times_{U, t} R \times_{s, U} U') = R'
$$
and for this we can use the formula
$((u'_1, r_1), (u'_2, r_2)) \mapsto (u'_1, r_1 \circ r_2^{-1}, u'_2)$
in scheme valued points. To check the cocycle condition, use the
cocycle condition on $\alpha'$; horrible details omitted.
This gives a quasi-coherent module $\mathcal{F}$ on $U$.
Then finally you have to use the cocycle condition on $\alpha'$
again to show that you can descend $\alpha'$ to get
$\alpha : t^*\mathcal{F} \to s^*\mathcal{F}$ on $R$. Lot's of
details omitted.
\end{proof}



\section{Preliminaries}
\label{section-preliminaries}

\noindent
Some definitions and lemmas.

\begin{definition}
\label{definition-representations}
Let $(U, R, s, t, c)$ be a groupoid scheme. The
{\it category of vector bundles on $(U, R, s, t, c)$} denoted
$\textit{Vect}(U, R, s, t, c)$ is the full subcategory of
$\QCoh(U, R, s, t, c)$ consisting of pairs $(\mathcal{F}, \alpha)$
with $\mathcal{F}$ finite locally free.
\end{definition}

\begin{lemma}
\label{lemma-vector-bundles-on-groupoid}
In the situation of Definition \ref{definition-representations}
the category $\textit{Vect}(U, R, s, t, c)$ is an additive symmetric
monoidal category where every object has a dual.
\end{lemma}

\begin{proof}
Omitted.
\end{proof}

\begin{definition}
\label{definition-affine-fpqc-gerbe}
Let $k$ be a field. Let $(U, R, s, t, c)$ be a groupoid scheme over $k$.
We say $(U, R, s, t, c)$ is {\it AFG} if the following conditions hold:
\begin{enumerate}
\item $U$ is not empty,
\item $U$ and $R$ are affine,
\item $(s, t) : R \to U \times_{\Spec(k)} U$ is flat and surjective.
\end{enumerate}
\end{definition}

\noindent
The letters AFG stand for affine fpqc gerbe, because the conditions guarantee
that the quotient stack $U/G$ in the fpqc topology is a gerbe over $\Spec(k)$.

\begin{lemma}
\label{lemma-vector-bundles-on-groupoid}
Let $k$ be a field. Let $(U, R, s, t, c)$ be AFG. The category
$\textit{Vect}(U, R, s, t, c)$ is Tannakian and the forgetful
functor $F : \textit{Vect}(U, R, s, t, c) \to \textit{Vect}(U)$
is a fibre functor.
\end{lemma}

\begin{proof}
Omitted. We may want to move this lemma later.
\end{proof}

\begin{lemma}
\label{lemma-restrict}
Let $k$ be a field. Let $(U, R, s, t, c)$ be AFG.
For any nonempty affine scheme $U'$ and morphism $U' \to U$ the
restriction $(U', R', s', t', c')$ is AFG.
\end{lemma}

\begin{proof}
Omitted.
\end{proof}

\begin{lemma}
\label{lemma-restrict-equivalences}
In the situation of Lemma \ref{lemma-restrict} the pullback functor
$\QCoh(U, R, s, t, c) \to \QCoh(U', R', s', t', c')$ is an equivalence
and determines an equivalence
$\textit{Vect}(U, R, s, t, c) \to \textit{Vect}(U', R', s', t', c')$.
\end{lemma}

\begin{proof}
Omitted.
\end{proof}

\begin{lemma}
\label{lemma-vect-is-ft-qcoh}
Let $k$ be a field. Let $(U, R, s, t, c)$ be AFG.
Then $\textit{Vect}(U, R, s, t, c)$ is equal to the
full subcategory of $\QCoh(U, R, s, t, c)$ consisting of
finite type quasi-coherent modules on $(U, R, s, t, c)$.
\end{lemma}

\begin{proof}
Omitted. Hint:
Follows by reduction to the case where $U$ is the spectrum of
a field which we can do by Lemma \ref{lemma-restrict-equivalences}.
\end{proof}

\begin{lemma}
\label{lemma-colimit}
Let $k$ be a field. Let $(U, R, s, t, c)$ be AFG.
Then we have
$\QCoh(U, R, s, t, c) = \text{Ind}(\textit{Vect}(U, R, s, t, c))$.
\end{lemma}

\begin{proof}
Sketch: First by Lemma \ref{lemma-restrict-equivalences}
we may assume $U = \Spec(K)$ where $K/k$ is a field extension.
In this case the lemma follows from Lemma \ref{lemma-vect-is-ft-qcoh}
and Groupoids, Lemma \ref{groupoids-lemma-colimit-finite-type}.
\end{proof}





\section{Going back}
\label{section-going-back}

\noindent
Let $k$ be a field. Let $(U, R, s, t, c)$ be AFG. Consider the fibre functor
$$
F : \textit{Vect}(U, R, s, t, c) \to \textit{Vect}(U)
$$
of Lemma \ref{lemma-vector-bundles-on-groupoid}. For any scheme
$U'$ over $U$ we can consider the composition
$$
F' : \textit{Vect}(U, R, s, t, c) \to \textit{Vect}(U')
$$
of $F$ with the pullback functor $\textit{Vect}(U) \to \textit{Vect}(U')$.
In particular we have two functors
$$
F_1, F_2 :
\textit{Vect}(U, R, s, t, c)
\to
\textit{Vect}(U \times_{\Spec(k)} U)
$$
corresponding to the two projections
$\text{pr}_i : U \times_{\Spec(k)} U \to U$. We consider the functor
$$
\mathcal{R} :
\textit{Schemes}/U \times_{\Spec(k)} U \longrightarrow \textit{Sets}
$$
sending a scheme $T/U \times_{\Spec(k)} U$ to the set of isomorphisms
of $k$-linear symmetric monoidal functors from $F_1|T$ to $F_2|_T$.
Let us denote $h_R$ the representable functor
on $\textit{Schemes}/U \times_{\Spec(k)} U$ given by $R$ viewed as
a scheme over $U \times_{\Spec(k)} U$ via $(s, t)$.
We claim there is a transformation of functors
$$
h_R \longrightarrow \mathcal{R}
$$
Namely, given $r : T \to R$ and an object $V = (\mathcal{F}, \alpha)$ of
$\textit{Vect}(U, R, s, t, c)$ the map $\alpha$ induces an isomorphism
$$
(F_1|_T)(V) = (s \circ r)^*\mathcal{F} = r^*s^*\mathcal{F}
\xrightarrow{r^*\alpha^{-1}}
r^*t^*\mathcal{F} = (t \circ r)^*\mathcal{F} = (F_2|_T)(V)
$$
functorial in $V$ and compatible with tensor products.

\begin{lemma}
\label{lemma-equivalence}
The transformation $h_R \longrightarrow \mathcal{R}$
described above is an isomorphism.
\end{lemma}

\begin{proof}
Let $T$ be a scheme and let
\end{proof}





\section{Tried something}
\label{section-try}

\medskip\noindent
OK, so I am wondering if we can rely as much as possible on the
(generalization of) the argument by Yifei in the writeup on C1.
Let me try this with the proof of the main question in C4.
Here is my attempt --- of course I am very happy if there are
better ways to do this.

\medskip\noindent
Given is a groupoid scheme $(U, R, s, t, c)$ over a field $k$ with
$U$, $R$ affine and $(s, t)$ surjective and flat.

\medskip\noindent
Write $U = Spec(A_U)$ and $R = Spec(A_R)$.

\medskip\noindent
Set $\mathcal{C} = Vect(U, R, s, t, c)$ as a $k$-linear tensor category.
Consider the obvious fibre functor $F : \mathcal{C} \to Mod_{A_U}$.
Of course for $V$ in $\mathcal{C}$ the module $F(V)$ is finite locally
free over $A_U$.

\medskip\noindent
Let $A$ be as in 1.1.3 of the file C1.pdf using $F$ but with tensor
products over $k$. In other words, we set
$$
A = \text{colim}_{V \to W} F(V) \otimes_k F(W)^\vee
$$
viewed as an $A_U \otimes_k A_U$-algebra. Note that $F(W)^\vee$ is the
$A_U$-linear dual here and not the $k$-linear dual. I think this is
still commutative, but I did not check this at all!

\medskip\noindent
GOAL: construct a canonical isomorphism $A = A_R$ as
$A_U \otimes_k A_U$-algebras.

\medskip\noindent
The first point to make is that there is a canonical map
$$
A \longrightarrow A_R
$$
Namely, let $V \to W$ be a morphism in $\mathcal{C}$.
I am going to factor the map $F(V) \otimes_k F(W)^\vee \to A_R$
through a map $F(V) \otimes_k F(V)^\vee \to A_R$, so I can just
work with $V$. Write $V = (\mathcal{F}, \alpha)$ where $\mathcal{F}$
is a finite locally free $\mathcal{O}_U$-module and
$\alpha : t^*\mathcal{F} \to s^*\mathcal{F}$ is the assumed
isomorphsm on $R$. Then we have
$$
F(V) \otimes_k F(V)^\vee =
\Gamma(U, \mathcal{F}) \otimes_k \Gamma(U, \mathcal{F}^\vee)
$$
Now an element of the right hand side gives by pulling back
by $s$ and $t$ on the first and second factor
(we may have to switch this) a global section of
$$
s^*\mathcal{F} \otimes_{\mathcal{O}_R} t^*\mathcal{F}^\vee
\xrightarrow{1 \otimes \alpha}
s^*(\mathcal{F} \otimes_{\mathcal{O}_U} \mathcal{F}^\vee)
\xrightarrow{evaluation}
\mathcal{O}_R
$$
and hence we get a section of the structure sheaf of $R$, i.e.,
an element of $A_R$. Then you have to check this is compatible
with the transition mappings.

\medskip\noindent
Denote $F' : QCoh(U, R, s, t, c) \to Mod_{A_U}$ the obvious
extension of $F$ to the category of all quasi-coherent modules
on the groupoid. By Tag 07TR there is an object
$V' = (t_*\mathcal{O}_R, \alpha)$ in $QCoh(U, R, s, t, c)$
such that $F'(V') = A_R$. Evaluation at the neutral section
$e : U \to R$ should give a map $V' \to 1$ in $\QCoh(U, R, s, t, c)$.
Then the construction above should still give a map
$$
F'(V') \otimes_k F(1)^vee = A_R \otimes_k A_U \longrightarrow A_R
$$
which should show that our map $A \to A_R$ is surjective because
we know that $V'$ can be written as a filtered colimit of objects
in $\mathcal{C}$. Lot's of details omitted; please ask me.

\medskip\noindent
As Yifei pointed out the tricky part should be to show that $A \to A_R$
is injective. I think to do this we interpret an element $a$ of $A$
as an $A_U \otimes_k A_U$-linear endo $t_a$ of the tensor functor
$F \otimes_k A_U$. If $a$ maps to zero in $A_R$, then I think $t_a$
sort of restricts to zero when you go to $F \otimes_{A_U} A_R$.
My idea is that we should look at the action of $t_a$ on $F'(V')$
with $V'$ as above to see that $a$ would have to be zero,
but I didn't try too hard.





\input{chapters}

\bibliography{my}
\bibliographystyle{amsalpha}

\end{document}
